\subsection{k-grid study}
	At the beginning of the calculations we demonstrated, that for the evaluation of the bulk HgTe the most economic k-grid setting for dividing the reciprocal space is 8x8x8. Note that for higher k-grids, which means smaller steps of discretization, more CPU time is required but would not harm the calculations. 
	
	The same study was performed on different slab thicknesses. Those results showed that the total energy as a function of k-grid steps was oscillating before it finally converged like the study did for the bulk. This oscillation has its origins in the broken translation symmetry in the direction perpendicular to the surface. The thinner the slabs is, the more apparent this side effect becomes because it affects more atoms per unit cell. 
%	Beginning with the k-grid outcome in figure \ref{k_grid_study}, the number points for the bulk must be at least 8, while a higher number is not necessary but doesn't harm the calculations either. 
%	In comparison to the $1/e^x$ like behavior of the bulk k-grid in \ref{k_grid_1}, the k-grid for the slabs does oscillate until the total energy converges at approximately 24 points in x and y direction is enough for 4 and 8 layers, see \ref{k_grid_2}, \ref{k_grid_3}. That oscillation calms down the larger the slabs become, hence the total energy converges earlier for 16 layers, see \ref{k_grid_4}, and the k-grid could be chosen smaller than for the thinner slabs. The reason for the bigger resemblance of thicker slabs to the bulk is, that the more layers the slab has, the more bulk-like it will be, because of the expansion in the third direction. This also applies to the band structure. 
%	
\subsection{Lattice constant study}	
	In subsection \ref{lattice_k-grid} we performed the study of the lattice constant for bulk HgTe. We showed that the minimum value of energy is reached at the lattice constant $a= 6.685 \,\unit{\AA}$. For comparison the experimental value at room temperature is $a=6.46152 \,\unit{\AA}$ and for LDA calculations, which was explained in subsection \ref{xc-func}, the value is $a= 6.346 \,\unit{\AA}$ (data from \cite{lattice_constant}).
%	For the investigation of the lattice constant with lowest rest energy, a high enough k-grid must be chosen, otherwise it would not necessarily converge, see figure \ref{kgrid_lattice_constant}. The outcome illustrated in figure \ref{lattice_constant} shows a minimum around 6.685 $\unit{\AA}$. For comparison the experimental value in room temperature is around 6.46152 $\unit{\AA}$, while DFT-LDA calculations come to 6.346 $\unit{\AA}$ and ab initio GGA, which is the topic of the method used here, are 6.656 $\unit{\AA}$ (all data taken from \cite{lattice_constant}). This means the evaluated lattice constant is units of hundredth Anström too big, thus the calculations brought a quite accurate result.
	
\subsection{Bulk band structure}
	We calculated the bulk band structure for HgTe with and without spin-orbit coupling for regarding the impact spin-orbit interactions have on mercury telluride. 
	%	Both plots are replica of result for the ones without SOC found in the literature \cite{bulk_bs}. 
	
	It seems that, because HgTe is made out of heavy elements, the calculations in which spin-orbit coupling is included are giving  physically meaningful results. Due to the spin-orbit interaction the original spin-degenerated bands are splitting up, as shown in figure \ref{bulk_band_structure}. 
	It is well known that for zinc-blende semiconductors the band splitting leads to a heavy hole band, a light gap band and a double degenerated conduction band near the $\Gamma$ point \cite{HgTe_structure_001} which is reproduced by our calculations. 
%	The band structure without using the SOC agree with the ones that can be found in several papers like in \cite {bulk_bs}. But the calculations only bring physically accurate results if the SOC is included and always differ from the band structure of non included SOC. Figure \ref{bulk_band_structure} mostly shows the band splitting due to the spin momentum locking which was explained in the theoretical part \ref{topological_insulator} and \ref{chapter_soc}.
	
\subsection{PBBS and the band structure of the slabs}
	In the main part we made the projected bulk band structure which represents the energy dispersion relation for electrons in HgTe. Therefore we calculated the dispersion relation for different $k_z$ in order to study slices of the first 3D Brillouin zone. In our case we concentrate on the (001) direction. As mentioned before in subsection \ref{surface_modeling}, this the growing direction of the slabs on which we want to analyze the potentially topological surface states. The plot for the projected bulk band structure in $k_z$ direction is shown in figure \ref{bulk_band_structure_generation}.  The grey area correlates to the energy values an electron can adopt if one is looking on the bulk band structure from $k_z$ direction. Note that this only gives information about the bulk. 
	
	Therefore we calculated the band structures for the slabs with different thickness and terminations at the surfaces. Note that the Fermi energy was set to zero and that we only regarded ideal surfaces. Of course we see the band splitting caused by the spin-orbit interaction, seen in the bulk band structure, also in the slab surface bands.
	
	After we superimposed the band structures of the slabs and the PBBS, we noticed that the slab bands appear in the white area of the PBBS as well as in the grey area, which belongs to the bulk. 
	The bands of the slabs, which are in the grey region, are bulk energy bands while the slab bands which cross the white area, the band gap of the bulk, are the surface states. The latter can be either trivial or topological surface states, which we can distinguish by looking where those bands come from and where they go to. A band can only represent a topological surface state if the band comes from the valence/conduction band and goes to the conduction/valence band. 
	Additionally one can count how many times the dispersion line crosses through the Fermi level. If the number of crossings is odd, then this energy band is a topological surface state \cite{Bansil}.
	
	By looking at figure \ref{bulk+surface_even_layers}, \ref{bulk+surface_odd_layers_Te} and \ref{bulk+surface_odd_layers_Hg} we notice that the slab band structure is very sensitive under variation of thickness and terminations. It stands out, that the number of bands rises as the slabs are becoming thicker. In the plots for 16 and 17 layers we find energy bands which lie entirely within the PBBS. These represent the bulk states. On the contrary the bands lying completely in the band gap are the surface states which can be seen clearly in figure \ref{bulk+surface_even_layers}, \ref{bulk+surface_odd_layers_Te} and \ref{bulk+surface_odd_layers_Hg} in (e) and (f). 
	
	Since we turn our attention particularly to the evolution of the topological surface states, we observe that the structure of the dispersion relation differs much for different surface terminations. 
	Regarding the slabs with one surface passivated by hydrogens, then we see that the energy bands corresponding to the surface states are shifted away from the Fermi level \cite{top_surf_states}. In the plots for clean terminations one can observe possible candidates for a Dirac cone.
	
	By counting all crossings of the dispersion energy through the Fermi level we only discover trivial surface states. This is provoked by the dangling bonds at the crystal's surface. 
	In order to see topological surface states we could just use strained HgTe like in \cite{HgTe_structure_001} or use the c(2x2)-type reconstruction like in \cite{top_surf_states}. In the latter case they studied Te-terminated slabs and found Dirac cones at the $\Gamma$ point once strain was applied and semi-infinite slabs were studied. For symmetric Hg-terminations they found Dirac cones at $\overline{\text{K}}$ and $\overline{\text{J}}$ in (001) direction in case of clean surfaces. 
	The hydrogens effects the isotropic Dirac cone in a way that they become anisotropic cones.  
	
	The reconstruction of the surfaces, the simulation of strains and the simulation of semi-infinite surfaces is beyond this bachelor thesis. 
%	Regarding the left column in the figures \ref{bulk+surface_even_layers}, \ref{bulk+surface_odd_layers_Te} and \ref{bulk+surface_odd_layers_Hg}, there are four bands, means eight states, near the Fermi level which belong to the surface. The four upper states are unoccupied and separated from the bulk, while the lower states merge with the bulk at some point. Their shape depend on the terminations. For example in the Te-Hg termination plots \ref{bulk+surface_even_layers} on the left side, the band which crosses the Fermi level at the $\overline{\text{K}}$ point, can be found in the left column of the Hg termination \ref{bulk+surface_odd_layers_Hg}. In contrast, the other bands which traverse the $\overline{\text{K}}$ point below the Fermi level but still not in the bulk, are only found again in the Te-termination plots \ref{bulk+surface_odd_layers_Te} (information from \cite{top_surf_states}).
%
%	In the right part of those band structures, there can be seen, that part a part of those surface states near the Fermi level are shifted away from that level. The reason is the passivation of the danging bonds of the lower surface. Because just one of the two surfaces is occupied by hydrogens, just half of the bands are shifted. And also here conformities between the mono-atom termination and the Te-Hg termination can be found. 
%	
%	Regarding the $\overline{\Gamma}$ point, most small layered slabs do not have an s-like surface band at the Fermi level which would be typical for an inverted band structure in topological insulators \cite{2D_top_ins}, just in case of Te terminated 5 layer slabs, the Hg terminated 5 layer slab with hydrogens and in nearly all 16 or 17 layer slabs except the Hg terminated and hydrogen occupied 17 layer slab. This means the band gap does not contain s-like states at Fermi level in all other slabs. 
%	
%	The hydrogen occupation leads not only to the band shift but also to more space between the bands. The anisotrophy of the Dirac cone disappears and the occupied dangling bonds no longer support an conducting surface. 
	
%	\subsection{Density of states} 
%	As one can see in all dos plots \ref{dos_surface_even_layers}, \ref{dos_surface_odd_layers_Te} and \ref{dos_surface_odd_layers_Hg}, there is no band gap in which the dos would be zero but in the 5 and 9 layer slabs without hydrogens with Te termination the dos approaches zero and gives rise to assume a band gap. The Te terminated 17 layer slab without hydrogens there is a slump near the Fermi level which is missing in the Te-Hg termination because at the same place in the Hg terminated slab there is a climax in the dos. It seems, that for hydrogen occupied surfaces, the dos plots get smother and the peaks lower but, since above the Fermi level the dos is in general higher than in the ones for no hydrogens added slabs, it can be assumed that the availability of states in the conduction band is better than for no hydrogens. 
	
	
%	, only the thickest slabs, meaning containing 16 or 17 layers, without hydrogens having a surface band at the Fermi level.
	