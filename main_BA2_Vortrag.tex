\documentclass[english,ph, handout]{URbeamer}

\usepackage{babel}
\usepackage[utf8]{inputenc}
\usepackage[T1]{fontenc}
\usepackage{lmodern}
\usepackage{geometry}
\usepackage{graphicx}
\usepackage{amsmath}
\usepackage{multimedia}
\usepackage{setspace}
\usepackage{braket}
\usepackage[ugly]{units}
%\usepackage{placeins}
\usepackage[backend=bibtex, style=numeric-comp, sorting=nty]{biblatex}
\addbibresource{bib.bib}

\newcommand{\diff}{\mathrm{d}}

\setbeameroption{show notes}
\let\orignote\note
%\renewcommand{\note}[1]{\orignote{
%		\begin{minipage}{\textwidth}
%			#1
%		\end{minipage}
%	}}
\renewcommand{\note}[1]{\orignote{\scriptsize{
			\begin{singlespace}
				#1
			\end{singlespace}
			}}}

\title{Evolution of topological surface states of thin HgTe-films with film thickness}
\institute{Fakultät für Physik}
\subtitle{Bachelor thesis}
\author{Tamara Szecsey, \today}
\date{}

\begin{document}
	\frame[plain]{\titlepage \note{Hello, my name is Tamara Szecsey and I will now give you a talk about my bachelor thesis topic, the evolution of topological surface states of thin HgTe-films with film thickness.}}
	
	\begin{frame}{Table of contents}
		\tableofcontents 
		\note{First I will introduce the topic of this thesis, then, in the theory part, I explain characteristics of topological insulators, how to describe a crystal structure and the surfaces so that computational calculations can be performed, the densitiy functional theory and the spin-orbit coupling.
		In the third part I will talk about the results I got and if we can see topological surface states and finally I present the conclusions. }
	\end{frame}
	
	\section{Introduction}
	\begin{frame}{Introduction}
		HgTe is a II-VI semiconductor and a topological insulator (TI).\footfullcite{top_surf_states}
		\vspace{10px}
	\begin{columns} 
		\begin{column}<2->{0.35\linewidth} \centering
			\includegraphics[width=\linewidth]{extrabilder_fuer_vortrag/Introduction1.jpg}
		\end{column}
		\begin{column}<2->{0.23\linewidth} 
			conducting surface states
			
			\vspace{5px}
			insulating in the bulk
			
			\vspace{15pt}
		\end{column}\hfill
		\begin{column}<3->{0.5\linewidth} 
			TIs are interesting because:
			\begin{itemize}
				\item<4-> spin-momentum locking
				\item<5-> protected surface states
			\end{itemize}
		\end{column}\hfill
	\end{columns}
	
	\note{HgTe belongs to the class of mercury-based II-VI(6) semiconductors which appear in zinc-blende structure. It is additionally found to be a topological insulator.\\
	A topological insulator is a material which is insulating in its interior, also called the bulk of a material, and that develops conducting states at the surface.\\
	They are very attractive for potential applications because the spin and the momentum of the electrons at the conducting surfaces are locked. This means, the spin is always perpendicular to the direction of motion of those electrons.
	The spin-momentum locking makes sure that the surfaces are protected if time-reversal symmetry is preserved. In other words the surface states remain, even for non-magnetic impurities.\\}
	\begin{columns}
		\begin{column}<6->{\linewidth}
		Calculations were performed using density functional theory (DFT) through FHI-aims.
		\end{column}
	\end{columns}	
	\begin{columns}<7->
		\begin{column}{0.1\linewidth}
			\vspace{9px}
			$0.5 \unit{nm}~\approx$
		\end{column}
		\begin{column}{0.8\linewidth}
			\includegraphics[width=\linewidth]{extrabilder_fuer_vortrag/Introduction2.jpg}
		\end{column}
			\hspace{-15px} 
		\begin{column}{0.1\linewidth}
			\vspace{.2px}
			$\approx~2.7 \unit{nm}$
		\end{column}
	\end{columns}
	\note{The calculations I performed were ab-initio calculations and therefore I used the density functional theory with the exchange-correlation function method PBE through FHI-aims.\\}
	\note{HgTe was examined in thin films from approximately 0.5 nm to 2.7 nm, which are 4 to 17 layers of atoms. Surfaces can only be examined in one direction at once. Because otherwise it would went beyond the scope of a bachelor thesis, we chose just one direction of observance, namely the (001) direction.\\
	The main goal this thesis pursued was examining the evolution of topological surface states for different slab thicknesses. Therefore I will now explain the theory behind the realization of this examination.}
	\begin{block}<8->{}
		Main goal: Evolution of surface states in different slab thicknesses.
	\end{block}
\end{frame}

%%% Local Variables:
%%% mode: latex
%%% TeX-master: "main_BA2_Vortrag.tex"
%%% End:
	
	\section{Theory}
	\begin{frame}{Theory: Topological insulator}
	2005 Kane and Mele found another class of material: \\the topological insulator (TI).
	\\
	\begin{columns}
		\begin{column}<2->{0.33\linewidth}
			spin-orbit coupling
		\end{column}
		\hspace{-1cm}
		\begin{column}<3->{0.33\linewidth}
			$\rightarrow$ band inversion
		\end{column}
		\hspace{-1.2cm}
		\begin{column}<4->{0.33\linewidth}
			$\rightarrow$ Dirac cone
		\end{column}
	\end{columns}
	\begin{columns}
		\begin{column}<4->{.5\linewidth}
			\begin{figure}
				\includegraphics[width=\textwidth]{andere_bilder/band_structure_top_insulator}
			\end{figure}
		\end{column}
		\begin{column}{.5\linewidth}
			\begin{block}<5->{2D TI: }
			have quantum spin edge states found in graphene and HgTe quantum wells.
			\end{block}
			\begin{block}<6->{3D TI: }
			strong and weak ones \\
			HgTe is semi metal but under strain $\Gamma_6$ and $\Gamma_8$ bands can close up at Fermi level. 
			\end{block}
		\end{column}
	\end{columns}
	\note{Besides from insulators and conductors, Kane and Mele 2005 found another class of material which is the so-called topological insulator. Most of the unusual properties come from the including of the spin-orbit coupling. This produces the spin-momentum locking mentioned earlier and can lead to band inversion. The spin-momentum locking and time-reversal symmetry prevent backscattering, because for two opposite momentums, the spin is also opposite and the states interfere destructively.  In other words, the surface states are protected by time-reversal symmetry. If the surface states close up because of band inversion then a so-called Dirac cone arises like seen in this picture from wikipedia. Here we see the momentum plotted against the energy near Fermi level. The blue line is the valence band and the red one is the conduction band. The surface states in green have a spin locked to their momentum and are forming a Dirac cone because their bands are inverted. \\	
	In 2D there were found topological surface states in graphene and HgTe quantum wells which are called Quantum spin edge states based on the Quantum spin hall effect. These quantum wells have non-trivial topological surface states under a certain critical thickness but become trivial insulators as soon as they pass this critical thickness.
	3D topological insulators are distinguishable into strong and weak topological insulators where the number of dirac cones is odd for strong and even for weak ones. HgTe normally is a semi metal but under strain its $\Gamma_6$ and $\Gamma_8$ bands can close up at Fermi level, which means its surface states become topological.}
\end{frame}

\begin{frame}{Theory: Crystal and surface description part 1}

	\begin{block}{Translation vector: $\boldsymbol{R}_i = x_i \boldsymbol{a}_1 + y_i \boldsymbol{a}_2 + z_i \boldsymbol{a}_3$}
		contains basis for lattice forming a unit cell. \\
		Smallest primitive unit cell is the Wigner Seitz cell.
	\end{block}
	\begin{columns}
		\begin{column}<2->{0.3\linewidth}
			Bravais lattice  
		\end{column}
	\end{columns}
	\begin{columns}
		\begin{column}<3->{0.5\linewidth}
			fcc lattice with two atomic basis 
		\end{column}
		\hspace{-1cm}
		\begin{column}<4->{0.55\linewidth}
			$\rightarrow$ diamond or zinc-blende structure 
		\end{column}
	\end{columns}
	\hfill
	\begin{columns}<5->
		\begin{column}{.25\linewidth}
			\centering
			\includegraphics[width=\linewidth]{andere_bilder/zinc_blende}
		\end{column}
%		\hspace{-.7cm}
		\begin{column}{.25\linewidth} 
		\scriptsize{
			\centering
			\begin{tabular}{c c c c} 
				\hline
				& \textbf{x} & \textbf{y} & \textbf{z}\\ 
				\hline 
				\vspace{0.2cm} 
				$\boldsymbol{a}_1$ & $0$ & $\frac{a}{2}$ & $\frac{a}{2}$ \\
				\vspace{0.2cm}
				$\boldsymbol{a}_2$ & $\frac{a}{2}$ & $0$ & $\frac{a}{2}$ \\
				\vspace{0.2cm}
				$\boldsymbol{a}_3$ & $\frac{a}{2}$ & $\frac{a}{2}$ & $0$ \\
%			\end{tabular}	
%		\end{minipage}
%		\\
%		\begin{column}[c]{.33\linewidth}
%			\begin{tabular}{c c c c} 
%				\hline
%				& \textbf{x} & \textbf{y} & \textbf{z}\\ 
				\hline 
				\vspace{0.2cm}
				Te & $0$ & $0$ & $0$ \\
				\vspace{0.2cm}
				Hg & $\frac{a}{4}$ & $\frac{a}{4}$ & $\frac{a}{4}$
			\end{tabular}
		}
		\end{column}
		\begin{column}<6->{.5\linewidth}
			\begin{minipage}{\linewidth}
			Miller indices: $(hkl)$
			\begin{equation*}
			h : k : l = \frac{1}{x} : \frac{1}{y} : \frac{1}{z}
			\end{equation*}
			\end{minipage}
			\\
			\begin{minipage}{\linewidth}
			\centering
			\includegraphics[width=.5\linewidth]{extrabilder_fuer_vortrag/millersche_indizes_001}
			\end{minipage}
		\end{column}
	\end{columns}
	\note{The atoms of a crystal maintain a symmetric pattern that repeats itself in our case into the three spatial dimensions. In order to describe that crystalline structure, we make use of the translation vector R i which contains information about the coordinates x, y and z of an atom i and the primitive translation vectors a one, a two and a three that form a so-called unit cell. It is a primitive unit cell, if it contains the least possible amount of atoms. The smallest primitive unit cell is called Wigner Seitz cell and has one single atom in its center. \\
	The lattices which are generated by R i are called Bravais lattice and in three dimensional space there are 14 of them. The face-centered cubic (fcc) lattice with two atomic basis is the structure needed here. If the atoms are of the same species, then it is called a diamond structure, if the atoms are of a different kind of species, it's called a zinc-blende structure. The latter is the one for mercury telluride. Here you can see the basis of it and a picture of a cubic. This is the basis for the bulk calculations. \\
	Surfaces however are described by the Miller indices h, k and l which are related to numbers x, y and z, and if one multiplies the basis vector with those numbers, the basis vectors touch the surface. If the basis vector can never touch the surface, then the Miller index is set to zero. This is why this plane (picture) is the (001) plane. }
\end{frame}

\begin{frame}{Theory: Crystal and surface description part 2}
	Basis for (001) direction of zinc-blende structure: 
	\begin{columns}
		\hspace{-1cm}
		\begin{column}<2->{.25\linewidth}
			\includegraphics[width=1.2\linewidth]{andere_bilder/zinc_blende_45degree.jpg}
		\end{column}
		\hspace{-1cm}
		\begin{column}<2->{.2\linewidth}\scriptsize{
			\begin{tabular}{c c c c} 
				\hline
				& \textbf{x} & \textbf{y} & \textbf{z}\\ 
				\hline 
				\vspace{0.2cm} 
				$\boldsymbol{a}_1$ & $\frac{a}{\sqrt{2}}$  & $0$ & $0$ \\
				\vspace{0.2cm}
				$\boldsymbol{a}_2$ & $0$ & $\frac{a}{\sqrt{2}}$ & $0$ \\
				\vspace{0.2cm}
				$\boldsymbol{a}_3$ & $0$ & $0$ & $a$ 
			\end{tabular}
		}	
		\end{column}
		\begin{column}<2->{.3\linewidth}\scriptsize{
			\begin{tabular}{c c c c} 
				\hline
				& \textbf{x} & \textbf{y} & \textbf{z}\\ 
				\hline
				\vspace{0.2cm} 
				Te & $0$ & $0$ & $0$ \\
				\vspace{0.2cm}
				Hg & $\frac{a}{2\sqrt{2}}$ & $0$ & $\frac{a}{4}$ \\
				\vspace{0.2cm}
				Te & $\frac{a}{2\sqrt{2}}$ & $\frac{a}{2\sqrt{2}}$ & $\frac{a}{2}$ \\
				\vspace{0.2cm}
				Hg & $0$ & $\frac{a}{2\sqrt{2}}$ & $\frac{3a}{4}$
			\end{tabular}
		}
		\end{column}
	\end{columns}
	\begin{columns}
		\begin{column}<3->{.6\linewidth}
			Reciprocal lattice: $e^{i\boldsymbol{K}\boldsymbol{R}} = 1$\\
			$\boldsymbol{K}$ is reciprocal translation vector.\\
			First Brillouin zone (BZ) is Wigner Seitz cell for reciprocal space.\\
			Super symmetrical points:\vspace{-.3cm}
			\begin{align*}
			\overline{\Gamma}&= 0;&
			\overline{\text{J}} &= \frac{1}{2} \boldsymbol{b}_1 ;&
			\overline{\text{K}}&= \frac{1}{2} \boldsymbol{b}_1 + \frac{1}{2} \boldsymbol{b}_2 
			\end{align*}
		\end{column}
		\begin{column}<3->{.4\linewidth}
			\includegraphics[width=\linewidth]{andere_bilder/brillouin_zone_001_2}
		\end{column}
	\end{columns}
	\note{For calculations in the (001) direction, it is convention to rotate the unit cell by 45 degrees which gives us another basis for HgTe.\\
	As we say before, the energy bands are obtained by observing the momentum with respect to the energy. The best way to do so is, to use the reciprocal lattice which is the Fourier transform of the real lattice with (formular with e function). K is the reciprocal translation vector with reciprocal lattice vectors b1, b2 and b3. The first Brillouin zone is the Wigner Seitz cell for the reciprocal space. It contains so-called high symmetry points, in the picture one can see a $\Gamma$ in the origin. Every other high symmetry point is indicated by Latin letters. The points with a bar on top are the high symmetry points for the 2D Brillouin zone which we will use for the surface calculations, in concrete: Gamma bar, J bar, K bar.}
\end{frame}

\begin{frame}{Theory: Surface modeling}
	\begin{columns}
		\begin{column}{.86\linewidth}
			\begin{block}{Termination}
				Two surfaces with symmetric Hg-Hg, Te-Te terminations or antisymmetric Te-Hg terminations.\\
				Add hydrogen atoms in order to saturate dangling bonds. 
			\end{block}
			\begin{block}<2->{Number of layers}
				Odd for symmetric, even for antisymmetric termination.\\
				Atoms with same $z$ component are in the same layer.\\
				I studied 3 different thicknesses for each termination.				
			\end{block}
			\begin{block}<3->{Supercell approach}
				Infinite repetition only possible in all directions.\\
				In order to avoid interactions between slabs in $k_z$ direction: add vacuum space. 
			\end{block}
		\end{column}
		\begin{column}<3->{0.14\linewidth} 
			\centering
			\includegraphics[width=\linewidth]{andere_bilder/hgte_16layer_supercell_2.jpg} 
		\end{column} 
	\end{columns}
	\note{Now I will talk about what we have to consider for simulating the surfaces in (001) direction. At first, there are different possible terminations at the top and the bottom surfaces. Either the terminations are symmetric Hg-Hg or Te-Te, or they are antisymmetric with Te-Hg. I also performed calculations with additional hydrogen atoms at one surface in order to saturate the dangling bonds (the unsatisfied valences on immobilized atoms). \\
	The number of layers corresponds to the kind of termination, it is odd for symmetric and even for antisymmetric ones. 
	In our case, one layer contains every atom with the same z component.
	And I studied three different numbers of layers or thicknesses for each kind of termination. \\
	Our approach for simulating the surface in one direction contains an infinite repetition in the other two directions. Unfortunately FHI-aims can only repeat in all three directions. In order to avoid interactions between the slabs in k z direction and simulate a surface, one adds vacuum space to the unit cell. This is called the supercell approach. The supercell for 16 layers is illustrated in the picture at the right, where a1 and a2 are as explained before, but a3 is not only as long as the slab is thick, but also contains additional vacuum space.}
\end{frame}

\begin{frame}{Theory: Density functional theory (DFT)}	
	\begin{block}{First Hohenberg-Kohn Theorem:}
		$\rho(\boldsymbol{r})$ can be uniquely converted into the $V_\text{ext}(\boldsymbol{r})$
	\end{block}
	\begin{block}<2->{Second Hohenberg-Kohn Theorem:}
		Ground state energy can be expressed in a density functional: \vspace{-.35cm}
		\begin{equation*}
		E^{\text{HK}}[\rho(\boldsymbol{r}); v_{\text{ext}}(\boldsymbol{r})] = 
		T_0 [\rho] + V_\text{H} [\rho] + \int v_{\text{ext}}(\boldsymbol{r}) \rho(\boldsymbol{r}) \diff \boldsymbol{r} + E_{\text{xc}} [\rho]
		\end{equation*}
	\end{block}
	\begin{block}<3->{Kohn-Sham equation}
		Minimizing $E^{\text{HK}}$ $\rightarrow$ Kohn-Sham eigenfunctions $\phi_i(\boldsymbol{r})$: \vspace{-.35cm}
%		 and $\rho(\boldsymbol{r}) = \sum_i |\boldsymbol{p}hi_i(\boldsymbol{r})|^2$:
		\begin{equation*} 
		E_0 [\rho_0] = E_{\text{nucl}} + E_{\text{kin}} + E_\text{H} + E_{\text{xc}} 
		\end{equation*}
	\end{block}
	\begin{block}<4->{Perdew-Burke-Ernzerhof (PBE) functional} \vspace{-.15cm}
		\begin{equation*}
		E_{\text{xc}}^{\text{PBE}} = \int \diff^3 r \rho(\boldsymbol{r}) 
		\,\epsilon_{\text{xc}}^{\text{PBE}} (r_s (\boldsymbol{r}), s(\boldsymbol{r}), \zeta(\boldsymbol{r}))
		\end{equation*}
	\end{block}
	\note{The density functional theory was first introduced by
		W.Kohn et al. They found that the ground-state energy of a quantum
		mechanical system can be represented by a density functional.
		For wave function of a many-body Schrödinger equation with an Hamiltonian containing the kinetic energy, the Coulomb potential between the particles, in our case electrons, and the external Potential which is the nuclei, one can define a one-body ground states density rho. The first Hohenberg-Kohn Theorem then says that rho can be uniquely converted into the external potential and vice versa. (In other words, for one given V ext, there exists specifically one rho) \\
		The second Hohenberg-Kohn Theorem says that the ground state energy of such a many-body system can be expressed in a density functional. T zero thereby is the kinetic energy for a system of non-interacting electrons, the rest builds the Hohenberg-Kohn potential which contains the Hartree potenial and the exchange-correlation function. Note that the small v ext is the external potential just for one electron but contains all protons of the nucleus.\\
		By minimzing this density functional we get the Kohn-Sham eigenfunctions phi i and the Kohn-Sham equation.	The exchange correlation functional can not be determined exactly, so there are several approximations been done. One of them is the Perdew-Burke-Ernzerhof functional, short: PBE. It is a semilocal-density functional,
		which does not only dependent on the density at a certain position. This means it is  better for big molecules or systems in comparison to the LDA functional, which is only local (as the name local density approach indicates). }
\end{frame}

\begin{frame}{Theory: Spin-orbit coupling}
	\begin{columns}<2->
		\begin{column}{.5\linewidth}
			Dirac equation:
		\end{column}\hspace{-3.3cm}
		\begin{column}{.7\linewidth}
			\begin{equation*}
			\left(
			c \boldsymbol\alpha \cdot \boldsymbol{p} + c^2(\beta - 1) + V
			\right) \Psi = \epsilon \Psi
			\end{equation*}
		\end{column}
	\end{columns}
	\begin{columns}<3->
		\begin{column}{.6\linewidth}
			Separate $\Psi$ into two spinors:
		\end{column}\hspace{-5cm}
		\begin{column}{.4\linewidth}
			\begin{equation*}
			\Psi = 
			\begin{pmatrix}
			\Psi_\text{L} \\ \Psi_\text{S}
			\end{pmatrix}
			\end{equation*}	
		\end{column}
	\end{columns}
	\begin{columns}<4->
		\begin{column}{\linewidth}
			Substitute and dissolve:
			\begin{align*}
			\left(
			\boldsymbol{p} \frac{c^2}{2c^2 + \epsilon - V} \boldsymbol{p} 
			+ i \boldsymbol{p} \frac{c^2}{2c^2 + \epsilon - V} \times \boldsymbol{p} \cdot \boldsymbol{\sigma} 
			+ V 
			\right) \Psi_\text{L} 
			&= \epsilon \Psi_\text{L}	
			\end{align*}
		\end{column}
	\end{columns}
	\vspace{.3cm}
	\begin{columns}<6->
		\begin{column}{\linewidth}
			First part with $V$ is relativistic Schrödinger equation.
		\end{column}
	\end{columns}
	\begin{columns}<7->
		\begin{column}{\linewidth}
			Expanding in $\frac{\epsilon - V}{2c^2}$ in zeroth order gives non-relativistic Schrödinger equation. 
		\end{column}
	\end{columns}
	\begin{columns}<8->
		\begin{column}{\linewidth}
			Expanding just second part in first order gives:
			\begin{equation*}
			V_{\text{SOC}} = \frac{i}{4c^2} \boldsymbol{p} V \times \boldsymbol{p} \cdot \boldsymbol{\sigma}
			\end{equation*}
		\end{column}
	\end{columns}
	\note{ The spin-orbit interaction is the interaction between a particles spin and its motion. The Dirac equation explains why it should be included in the Schrödinger equation.
	Here we see the Dirac equation for electrons, where alpha contains in the top at right and in the bottom at left a sigma vector with all three Pauli matrices, and beta is the matrix containing the 2D basis matrix in the left upper part and minus 2D basis matrix in the right lower part. p is the momentum. 
	Then we separate the wave function Psi into two spinors, large for the one that positive energies, means it represents the electrons, and is dominant for the non-relativistic limit, and small for negative energies, means positrons, which disappear in the non-relativistic limit. By substituting this into the Dirac equation and dissolve the two coupled equations we get this Schrödinger like equation. \\
	The first part together with the V is the relativistic Schrödinger equation. By expanding it in epsilon minus V over 2 times c squared in zeroth order and then sets epsilon to zero, it gives us the non-relativistic Schrödinger equation. 
	By expanding only the second part of that equation above in first order, one gets the spin-orbit coupling potential. \\
	If a structure contains heavy elements, like HgTe, it is necessary to include this potential since spin-orbit interaction also affects valence electrons. This is not the case for light elements. 
	The SOC calculations are performed by FHI-aims after the self consistency cycle converged. 
	In the band structure it can cause band splitting on degenerated bands. }
\end{frame}

%%% Local Variables:
%%% mode: latex
%%% TeX-master: "main_BA2_Vortrag.tex"
%%% End:
	
	\section{Results}
	\begin{frame}{lattice constant and k-grid study part 1}
	\begin{columns}
		\begin{column}{0.48\linewidth}
			\centering
			\includegraphics[width=\linewidth]{andere_bilder/plot_energies_hgte_bulk_all_kgrid_in_one.pdf}
			\\
			Lattice constant $a$ with respect to the total energy, calculated with 4 different k-grids. 
		\end{column}
		\begin{column}{0.48\linewidth}
			\centering
			\includegraphics[width=\linewidth]{andere_bilder/lattice_constant_study_spin_none_no_soc.pdf}
			\\
			Lattice constant $a$ with respect to the total energy. The lowest energy was found for 6.685\,$\unit{\AA}$. %\vspace{12.5pt}
		\end{column}
	\end{columns}
	\note{ Let's now turn to the results. First of all I like to mention, that the input data for calculations performed by FHI-aims are the geometry.in and the control.in. The geometry.in contains the coordinates of the lattice vectors and of the atoms. The control.in contains physical settings like the exchange-correlation method, the spin treatment and the relativistic effects. It also contains convergence criteria of the self convergence cycle, the k-grid settings and Informations about all atomic species represented in the geometry.in. \\
	Before starting the calculations for the band structure, it is recommended to do a
	lattice constant study and a k-grid study. So let's have a look at the plot at the left. Here we see the lattice constant with respect to the total energy of the bulk calculated with different k-grids. As we can see the k-grid 3 3 3 gives no good results. The plots for 8 to 16 kgrid overlays, so 8x8x8 seems to be enough to bring physically correct results. In the right picture we see the lattice constant plot for one specific kgrid configuration. The energetically most stable constitution of a crystal structure is given by the lattice constant $a$ with the smallest total system energy, in this case, I found 6.685 Angström for the lowest energy.}
\end{frame}

\begin{frame}{lattice constant and k-grid study part 2}
	\scriptsize{
	Total energy $E [\unit{eV}] - E_0$ (with $E_0$ the energy offset) as a function of k-grid spacing.} \vspace{.1cm}
	\begin{columns}
		\begin{column}{.4\linewidth}
%			\centering
			\scriptsize{
				Bulk: %with $E_0= -741008 \,\unit{eV}$
			}\\ 
			\includegraphics[width=\linewidth]{andere_bilder/kgrid_bulk.pdf}
		\end{column}
		\begin{column}{.4\linewidth}
			\scriptsize{
				Slab with 4 layers: %with $E_0= -1482047 \,\unit{eV}$
			}\\
			\includegraphics[width=\linewidth]{andere_bilder/kgrid_1x1x4_layers.pdf}
		\end{column}
		\end{columns}
		\begin{columns}
		\begin{column}{.4\linewidth}
			\scriptsize{
				Slab with 8 layers:% with $E_0= -2964065 \,\unit{eV}$
			}\\
			\includegraphics[width=\linewidth]{andere_bilder/kgrid_1x1x8_layers.pdf}
		\end{column}
		\begin{column}{.4\linewidth}
			\scriptsize{
				Slab with 16 layers: %with $E_0= -5928101 \,\unit{eV}$
			} \\
			\includegraphics[width=\linewidth]{andere_bilder/kgrid_1x1x16_layers.pdf}
		\end{column}
	\end{columns}
	\note{ This lattice constant then was used for a detailed kgrid study of the bulk and slabs with 4, 8 and 16 layers. Note that for the slabs, k z was set to 1 and that E 0 is just a number different for every plot which was chosen for smaller numbers in the y axis. Since a higher kgrid recommends more computational effort, the most economic kgrid directly after the it has converged to a certain energy. That number was nearly the same for the bulk and the all slabs, namely 12. The oscillation in the plots for the slabs comes from the translation symmetry break in k z direction and therefore it is no wonder that this oscillation becomes smaller for thicker slabs.
	The kgrid used for the following calculations was 24 since higher kgrid gives smoother bands and the additional computational effort was small.}
\end{frame}

\begin{frame}[fragile]{Projected bulk band structure (PBBS)}
	Making 40 slices in 2D Brillouin zone of the bulk from $k_z=0$ to $k_z=k_{z,\text{max}}$ where $k_{z,\text{max}}=\frac{\pi}{a}$. Input example:
	\vspace{-.3cm}
	\begin{columns} 
		\begin{column}<2->{\linewidth}\scriptsize{
			\begin{verbatim}
			output band 0.5   0.0   0.01175   0.0   0.0   0.01175    80  J     Gamma
			output band 0.0   0.0   0.01175   0.5   0.5   0.01175    80  Gamma K
			output band 0.5   0.5   0.01175   0.5   0.0   0.01175    80  K     J
			\end{verbatim} }
		\end{column}
	\end{columns}
	\begin{columns}<3->
		\begin{column}{.165\linewidth} \scriptsize{
				$k_z=0$}
		\end{column} \hspace{-.5cm}
		\begin{column}{.33\linewidth}
%		\begin{figure}[c]{\linewidth}
			\centering
			\includegraphics[width=\linewidth]{andere_bilder/0_bulk_-12_10.pdf}
%			\caption{PBBS for $k_z$ is equal to zero.}
%		\end{figure}
		\end{column}
		\begin{column}{.33\linewidth}
%		\begin{figure}[c]{\linewidth}
			\centering
			\includegraphics[width=\linewidth]{andere_bilder/4_bulk_-12_10.pdf}
%			\caption{Ensemble for PBBS for $k_z$ having values from 0 to $1/10\cdot k_{z,\text{max}}$}
%		\end{figure}
		\end{column}
		\begin{column}{.165\linewidth} \scriptsize{
			${k_z=0}$ to ${1/10\cdot k_{z,\text{max}}}$ }
		\end{column}
	\end{columns}
	\begin{columns}
		\begin{column}{.165\linewidth} \scriptsize{
				${k_z=0}$ to ${1/4\cdot k_{z,\text{max}}}$ }
		\end{column} \hspace{-.5cm}
		\begin{column}{.33\linewidth}
%		\begin{figure}[c]{\linewidth}
			\centering
			\includegraphics[width=\linewidth]{andere_bilder/10_bulk_-12_10.pdf}
%			\caption{Ensemble for PBBS for $k_z$ having values from 0 to $1/4\cdot k_{z,\text{max}}$} 
%		\end{figure}
		\end{column}
		\begin{column}{.33\linewidth}
%		\begin{figure}[c]{\linewidth}
			\centering 
			\includegraphics[width=\linewidth]{andere_bilder/bulk_-12_10.pdf}
%			\caption{Whole PBBS for all the values of $k_z$ going from 0 to $k_{z,\text{max}}$} 
%		\end{figure}
		\end{column}
		\begin{column}{.165\linewidth} \scriptsize{
				${k_z=0}$ to $k_{z,\text{max}}$ }
		\end{column}
	\end{columns}	
	\note{ The bulk band structure alone is calculated along the high symmetry points in the 3D Billouin zone. But we want to compare it to the slab band structure with broken symmetry in k z direction. Therefore I used the so-called projected bulk band structure. This means I made a certain amount of slices of the 2D Brillouin zone of the bulk with k z from 0 to k z max which is pi over a. Here you can see an example of the input in the control.in  for kz is one fortieth of k z max. The first three numbers are the coordinates of the high symmetry points where the calculations begin, the fourth to sixth number are the coordinates of the point where the calculations end and 80 are the is the number of points which divides that segment of the path. These pictures show how more and more calculations are added until all results are superposed and the PBBS is complete. The last plot shows the bulk band structure regarded from k z point of view. The grey part is the bulk, the white part symbolizes the gaps. }
\end{frame}


\begin{frame}{PBBS with 4 and 5 layer slabs}
	\begin{columns}
		\begin{column}{.34\linewidth}
			\centering
			Te-Hg termination
		\end{column}
		\begin{column}{.34\linewidth}
			\centering
			Te-Te termination
		\end{column}
		\begin{column}{.34\linewidth}
			\centering
			Hg-Hg termination
		\end{column}
	\end{columns}
	\begin{columns}
		\begin{column}{.34\linewidth}
			\centering
			\includegraphics[width=\linewidth]{Te_and_Hg_termination/no_H_bulk+4_layers_no_dos_-2_2.pdf}
%			\caption{4 layers without hydrogens passivating the Te termination}
		\end{column}
		\begin{column}{.34\linewidth}
			\centering
			\includegraphics[width=\linewidth]{Te_termination/no_H_bulk+5_layers_no_dos_-2_2.pdf}
%			\caption{5 layers without hydrogens passivating one of the surfaces}
		\end{column}
		\begin{column}{.34\linewidth}
			\centering
			\includegraphics[width=\linewidth]{Hg_termination/no_H_bulk+5_layers_no_dos_-2_2.pdf}
%			\caption{5 layers without hydrogens passivating one of the surfaces}
		\end{column}
	\end{columns}
	\begin{columns}
		\begin{column}{.34\linewidth}
			\centering
			\includegraphics[width=\linewidth]{Te_and_Hg_termination/bulk+4_layers_no_dos_-2_2.pdf}
%			\caption{4 layers with hydrogens on the bottom passivating the Te surface terminations}
		\end{column}
		\begin{column}{.34\linewidth}
			\centering
			\includegraphics[width=\linewidth]{Te_termination/bulk+5_layers_no_dos_-2_2.pdf}
%			\caption{5 layers with hydrogens on the bottom passivating one of the Te surface terminations}
		\end{column}
		\begin{column}{.34\linewidth}
			\centering
			\includegraphics[width=\linewidth]{Hg_termination/bulk+5_layers_no_dos_-2_2.pdf}
%			\caption{5 layers with hydrogens on the bottom passivating one of the Hg surface terminations}
		\end{column}
	\end{columns}
	\vspace{.3cm}
	\footnotesize{
	First row: without hydrogens. \\Second row: with hydrogens passivating one surface.}
	\note{For comparing the PBBS with the slab band structure, I calculated them with the supercell approach and k z set to zero in the band output. The result is then superimposed to the PBBS. As you can see, these (point) are the plots for Te-Hg termination, these (point) for Te-Te and these for Hg-Hg termination, while in this row are all plots for calculations without hydrogens and these plots are all with hydrogens passivating one surface, in case of the Te-Hg termination, the Te surface is passivated. \\
	There are several things one notices while comparing these plots to each other. First of all by looking at them we notice that the slab band structure is very
	sensitive under variation of thickness and terminations: The number of bands rises for thicker slabs. Regarding the slabs with one surface passivated by hydrogens, we can see that the energy bands corresponding to the surface states are shifted away from the Fermi level. In the plots for clean terminations one can observe possible candidates for a Dirac cone (point at gamma point at Fermi level). Slab bands which are completely in the white area are the the surface states, band which are completely in the grey area are bulk bands. Only for 16 and 17 layers we can see bands which are completely in the grey and in the white area. \\
	The surface states which are candidates for topological states must come from valence band an got to the conduction band or vice versa. Additionally one
	can count how many times the dispersion line crosses through the Fermi level. If the
	number of crossings is odd, then this energy band is a topological surface state. When we count those crossings, we only find trivial surface states, which brings us to the conclusion.}
\end{frame}

\begin{frame}{PBBS with 8 and 9 layer slabs}
	\begin{columns}
		\begin{column}{.34\linewidth}
			\centering
			Te-Hg termination
		\end{column}
		\begin{column}{.34\linewidth}
			\centering
			Te-Te termination
		\end{column}
		\begin{column}{.34\linewidth}
			\centering
			Hg-Hg termination
		\end{column}
	\end{columns}
	\begin{columns}
		\begin{column}{.34\linewidth}
			\centering
			\includegraphics[width=\linewidth]{Te_and_Hg_termination/no_H_bulk+8_layers_no_dos_-2_2.pdf}
%			\caption{8 layers without hydrogens passivating the Te termination}
		\end{column}
		\begin{column}{.34\linewidth}
			\centering
			\includegraphics[width=\linewidth]{Te_termination/no_H_bulk+9_layers_no_dos_-2_2.pdf}
%			\caption{9 layers without hydrogens passivating one of the surfaces}
		\end{column}
		\begin{column}{.34\linewidth}
			\centering
			\includegraphics[width=\linewidth]{Hg_termination/no_H_bulk+9_layers_no_dos_-2_2.pdf}
%			\caption{9 layers without hydrogens passivating one of the surfaces}
		\end{column}
	\end{columns}
	\begin{columns}
		\begin{column}{.34\linewidth}
			\centering
			\includegraphics[width=\linewidth]{Te_and_Hg_termination/bulk+8_layers_no_dos_-2_2.pdf}
%			\caption{8 layers with hydrogens on the bottom passivating the Te surface terminations}
		\end{column}
		\begin{column}{.34\linewidth}
			\centering
			\includegraphics[width=\linewidth]{Te_termination/bulk+9_layers_no_dos_-2_2.pdf}
%			\caption{9 layers with hydrogens on the bottom passivating one of the Te surface terminations}
		\end{column}
		\begin{column}{.34\linewidth}
			\centering
			\includegraphics[width=\linewidth]{Hg_termination/bulk+9_layers_no_dos_-2_2.pdf}
%			\caption{9 layers with hydrogens on the bottom passivating one of the Hg surface terminations}
		\end{column}
	\end{columns}
	\vspace{.3cm}
	\footnotesize{
		First row: without hydrogens. \\Second row: with hydrogens passivating one surface.}
\end{frame}

\begin{frame}{PBBS with 16 and 17 layer slabs}
	\begin{columns}
		\begin{column}{.34\linewidth}
			\centering
			Te-Hg termination
		\end{column}
		\begin{column}{.34\linewidth}
			\centering
			Te-Te termination
		\end{column}
		\begin{column}{.34\linewidth}
			\centering
			Hg-Hg termination
		\end{column}
	\end{columns}
	\begin{columns}
		\begin{column}{.34\linewidth}
			\centering 
			\includegraphics[width=\linewidth]{Te_and_Hg_termination/no_H_bulk+16_layers_no_dos_-2_2.pdf}
%			\caption{16 layers without hydrogens passivating the Te termination} \label{}
		\end{column}
		\begin{column}{.34\linewidth}
			\centering 
			\includegraphics[width=\linewidth]{Te_termination/no_H_bulk+17_layers_no_dos_-2_2.pdf}
%			\caption{17 layers without hydrogens passivating one of the surfaces} 
		\end{column}
		\begin{column}{.34\linewidth}
			\centering 
			\includegraphics[width=\linewidth]{Hg_termination/no_H_bulk+17_layers_no_dos_-2_2.pdf}
%			\caption{17 layers without hydrogens passivating one of the surfaces} \label{}
		\end{column}
	\end{columns}
	\begin{columns}
		\begin{column}{.34\linewidth}
			\centering
			\includegraphics[width=\linewidth]{Te_and_Hg_termination/bulk+16_layers_no_dos_-2_2.pdf}
%			\caption{16 layers with hydrogens on the bottom passivating the Te surface terminations}
		\end{column}
		\begin{column}{.34\linewidth}
			\centering
			\includegraphics[width=\linewidth]{Te_termination/bulk+17_layers_no_dos_-2_2.pdf}
%			\caption{17 layers with hydrogens on the bottom passivating one of the Te surface terminations}
		\end{column}
		\begin{column}{.34\linewidth}
			\centering
			\includegraphics[width=\linewidth]{Hg_termination/bulk+17_layers_no_dos_-2_2.pdf}
%			\caption{17 layers with hydrogens on the bottom passivating one of the Hg surface terminations}
		\end{column}
	\end{columns}
	\vspace{.3cm}
	\footnotesize{
		First row: without hydrogens. \\Second row: with hydrogens passivating one surface.}
\end{frame}

%%% Local Variables:
%%% mode: latex
%%% TeX-master: "main_BA2_Vortrag.tex"
%%% End:
	
	\section{Conclusion}
	\begin{frame}{Conclusion}
	\begin{itemize}
		\item The k-grid study showed: at least 12 integration points. 
		\item<2-> Energy minimum at lattice constant $a = 6.685 \AA$. For comparison the experimental value at room temperature: $a = 6.46152 \AA$ and for
		LDA calculations: $a = 6.346 \AA$. \footcite{lattice_constant}
		\item<3-> We saw the impact of different thicknesses and terminations on the superimposed slab band structure on the PBBS.
		\item<4-> There are only trivial surface states because of the unsaturated dangling bonds.
		\item<5-> For observing topological surface states one should apply strain on HgTe \footcite{HgTe_structure_001} or do surface c(2x2)-type reconstruction with semi-infinite slabs \footfullcite{top_surf_states}.
	\end{itemize}
	\note{In the kgrid study we showed that for a physically correct result we have to use at least 12 integration points which is therefore also the most economic number of kgrids. \\
	Then we showed that for the pbe method in this case the minimum energy is reached for a lattice constant equal to 6.685 Angström. For comparison the experimental value at room temperature is 6.46152 Angstrom and for the local density approximation instead of PBE it is 6.346 Angström. \\
	We saw that different slab thicknesses and terminations had affect the slab band structure a lot. \\
	And by counting the crossings of the surface bands we found out that all of them are trivial surface states. The reason might be the unsaturated dangling bonds. \\
	In order to observe topological surface states one could apply strain on HgTe or do surface c 2times 2 type reconstruction with semi-infinite slabs. But this would beyond of scope for a bachelor thesis.\\}
\end{frame}

\begin{frame}
	\begin{block}{}
	\centering \Large{
	Thank you for your time!
	}
	\end{block}
	\note{Thank you for your attention!}
\end{frame}
%%% Local Variables:
%%% mode: latex
%%% TeX-master: "main_BA2_Vortrag.tex"
%%% End:
	
\end{document}

%%% Local Variables:
%%% mode: latex
%%% TeX-master: "main_BA2_Vortrag.tex"
%%% End: