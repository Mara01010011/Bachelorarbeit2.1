	All results were obtained by using FHI-aims \cite{aims}. 
	%Each calculation is been done in a separate folder.
	The input data are the files control.in and geometry.in.
	The geometry.in contains the lattice vectors' coordinates and the positions and types of the atom forming the basis. One example for the calculations of the bulk HgTe properties is:
		\begin{verbatim}
			# hgte lattice constant 6.685 AA
			lattice_vector	0.00000		3.34250		3.34250
			lattice_vector	3.34250		0.00000		3.34250
			lattice_vector	3.34250		3.34250		0.00000
			
			atom	0.00000		0.00000		0.00000 Te
			atom	1.67125		1.67125		1.67125 Hg
		\end{verbatim}
%	which is nothing else but the coordinates written in figure \ref{fcc}.
	
	The control.in contains all physical and computational settings for the calculation. The properties for the atoms also appear in the control.in, specifically the different types of basis sets settings.
%	are normally stored in the aims directory \texttt{SPECIES\_DEFAULTS} and have to be copied
	One can choose between \textit{light, tight} and \textit{really tight}, in this thesis we were using the \textit{light} settings.
%	But above that part, there are physical settings and the settings about the band structure output and the k-grid and about the convergence, for example for the k-grid study below:
	Among the setting one must consider is the exchange-correlation functional, the spin, the numerical settings for the self-consistency cycle convergence and the grid size of the reciprocal lattice. For example see what we used for the bulk HgTe calculations:
	\\
	\begin{minipage}[c]{0.45\linewidth}	\vspace{15pt}
		\begin{verbatim}
		# Physical settings
		xc pbe
		spin none
		relativistic atomic_zora scalar
		default_initial_moment 0
		#
		\end{verbatim} 
	\end{minipage}
	\begin{minipage}[c]{0.55\linewidth} \vspace{11pt}
		\begin{verbatim}
		
		# methode to be used, here pbe
		# spin treatment
		# include relativistic effects
		# initial spin for the atoms
		\end{verbatim}
	\end{minipage} 
	\begin{minipage}[c]{0.3\linewidth} \vspace{0.2cm}
		\begin{verbatim} 
			# SCF convergence
			sc_accuracy_rho		1E-4
			sc_accuracy_eev		1E-2
			sc_accuracy_etot		1E-5
			sc_iter_limit			100	
		\end{verbatim}
	\end{minipage} 
	\hfill
	\begin{minipage}[c]{0.7\linewidth} \vspace{0.15cm}
%		\vspace{8pt}
		\begin{verbatim}
		# convergence criteria of self convergence cycle
		# of change of density
		# of sum of orbital eigenvalues
		# total energy between two consecutive cycles
		# maximum of iterations
		\end{verbatim}
	\end{minipage} 
	\begin{minipage}[c]{0.3\linewidth}\vspace{0.2cm}
		\begin{verbatim}
			#
			# k_grid settings
			k_grid	8	8	8
		\end{verbatim}\vspace{8pt}
	\end{minipage}
	\hfill 
	\begin{minipage}[c]{0.7\linewidth}\vspace{28pt}
		\begin{verbatim}
		
		# k_x k_y k_z for the integration in k-space
		\end{verbatim}\vspace{8pt}
	\end{minipage} 

%	The spin was always set to none, otherwise, if it would be set to collinear, one would have to set \texttt{default\_initial\_moment} to \texttt{hund} or guessed it by hand.
	Spin none means that we deal with closed-shell calculations, in other words, the valence shell is regarded as filled \cite{closed_shell}. 
	The relativistic effects had to be taken into account since Hg and Te are both heavy elements and spin-orbit coupling, which is at the heart of the properties of topological insulators, can be considered as a low energy relativistic effect.
%	these effects, as explained in chapter \ref{topological_insulator}, are essential for the research of topological insulators.   
	
%%% Local Variables:
%%% mode: latex
%%% TeX-master: "main_BA2.0"
%%% End: