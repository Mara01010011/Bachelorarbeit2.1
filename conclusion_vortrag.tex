\begin{frame}{Conclusion}
	\begin{itemize}
		\item The k-grid study showed: at least 12 integration points. 
		\item<2-> Energy minimum at lattice constant $a = 6.685 \AA$. For comparison the experimental value at room temperature: $a = 6.46152 \AA$ and for
		LDA calculations: $a = 6.346 \AA$. \footcite{lattice_constant}
		\item<3-> We saw the impact of different thicknesses and terminations on the superimposed slab band structure on the PBBS.
		\item<4-> There are only trivial surface states because of the unsaturated dangling bonds.
		\item<5-> For observing topological surface states one should apply strain on HgTe \footcite{HgTe_structure_001} or do surface c(2x2)-type reconstruction with semi-infinite slabs \footfullcite{top_surf_states}.
	\end{itemize}
	\note{In the kgrid study we showed that for a physically correct result we have to use at least 12 integration points which is therefore also the most economic number of kgrids. \\
	Then we showed that for the pbe method in this case the minimum energy is reached for a lattice constant equal to 6.685 Angström. For comparison the experimental value at room temperature is 6.46152 Angstrom and for the local density approximation instead of PBE it is 6.346 Angström. \\
	We saw that different slab thicknesses and terminations had affect the slab band structure a lot. \\
	And by counting the crossings of the surface bands we found out that all of them are trivial surface states. The reason might be the unsaturated dangling bonds. \\
	In order to observe topological surface states one could apply strain on HgTe or do surface c 2times 2 type reconstruction with semi-infinite slabs. But this would beyond of scope for a bachelor thesis.\\}
\end{frame}

\begin{frame}
	\begin{block}{}
	\centering \Large{
	Thank you for your time!
	}
	\end{block}
	\note{Thank you for your attention!}
\end{frame}
%%% Local Variables:
%%% mode: latex
%%% TeX-master: "main_BA2_Vortrag.tex"
%%% End: